\PassOptionsToPackage{unicode=true}{hyperref} % options for packages loaded elsewhere
\PassOptionsToPackage{hyphens}{url}
\PassOptionsToPackage{dvipsnames,svgnames*,x11names*}{xcolor}
%
\documentclass[10pt,]{article}
\usepackage{lmodern}
\usepackage{amssymb,amsmath}
\usepackage{ifxetex,ifluatex}
\usepackage{fixltx2e} % provides \textsubscript
\ifnum 0\ifxetex 1\fi\ifluatex 1\fi=0 % if pdftex
  \usepackage[T1]{fontenc}
  \usepackage[utf8]{inputenc}
  \usepackage{textcomp} % provides euro and other symbols
\else % if luatex or xelatex
  \usepackage{unicode-math}
  \defaultfontfeatures{Ligatures=TeX,Scale=MatchLowercase}
\fi
% use upquote if available, for straight quotes in verbatim environments
\IfFileExists{upquote.sty}{\usepackage{upquote}}{}
% use microtype if available
\IfFileExists{microtype.sty}{%
\usepackage[]{microtype}
\UseMicrotypeSet[protrusion]{basicmath} % disable protrusion for tt fonts
}{}
\IfFileExists{parskip.sty}{%
\usepackage{parskip}
}{% else
\setlength{\parindent}{0pt}
\setlength{\parskip}{6pt plus 2pt minus 1pt}
}
\usepackage{xcolor}
\usepackage{hyperref}
\hypersetup{
            pdftitle={Drivers of internal displacement in Somalia: clarifying internal displacement as climate adaptation of what},
            colorlinks=true,
            linkcolor=blue,
            citecolor=blue,
            urlcolor=blue,
            breaklinks=true}
\urlstyle{same}  % don't use monospace font for urls
\usepackage{longtable,booktabs}
% Fix footnotes in tables (requires footnote package)
\IfFileExists{footnote.sty}{\usepackage{footnote}\makesavenoteenv{longtable}}{}
\usepackage{graphicx,grffile}
\makeatletter
\def\maxwidth{\ifdim\Gin@nat@width>\linewidth\linewidth\else\Gin@nat@width\fi}
\def\maxheight{\ifdim\Gin@nat@height>\textheight\textheight\else\Gin@nat@height\fi}
\makeatother
% Scale images if necessary, so that they will not overflow the page
% margins by default, and it is still possible to overwrite the defaults
% using explicit options in \includegraphics[width, height, ...]{}
\setkeys{Gin}{width=\maxwidth,height=\maxheight,keepaspectratio}
\setlength{\emergencystretch}{3em}  % prevent overfull lines
\providecommand{\tightlist}{%
  \setlength{\itemsep}{0pt}\setlength{\parskip}{0pt}}
\setcounter{secnumdepth}{5}
% Redefines (sub)paragraphs to behave more like sections
\ifx\paragraph\undefined\else
\let\oldparagraph\paragraph
\renewcommand{\paragraph}[1]{\oldparagraph{#1}\mbox{}}
\fi
\ifx\subparagraph\undefined\else
\let\oldsubparagraph\subparagraph
\renewcommand{\subparagraph}[1]{\oldsubparagraph{#1}\mbox{}}
\fi

% set default figure placement to htbp
\makeatletter
\def\fps@figure{htbp}
\makeatother

\usepackage{dcolumn}
\usepackage[T1]{fontenc}
\usepackage{inputenc}

\title{Drivers of internal displacement in Somalia: clarifying internal displacement as climate adaptation of what}
\author{Woi Sok Oh\textsuperscript{1,2}, Juan Rocha\textsuperscript{3}, Simon Levin\textsuperscript{2}\\
\small \textsuperscript{1}High Meadows Environmental Institute, Princeton University, Princeton, NJ 08544\\
\small \textsuperscript{2}Department of Ecology \& Evolutionary Biology, Princeton University, Princeton, NJ 08544\\
\small \textsuperscript{3}Stockholm Resilience Centre, Stockholm University, Kräftriket 2B, 10691 Stockholm}
\date{September 04, 2022}

\begin{document}
\maketitle
\begin{abstract}
Recent years have witnessed millions of global populations involuntarily migrating by external forces. In particular, existing studies increasingly use terms such as climate migration, climate refugee, or climate mobility to highlight population displacement driven by natural disasters or weather events. Some studies further suggest that migration/displacement is a process of climate adaptation. However, we lack evidence on what and how climatic drivers significantly displace people. There are many climatic components that we need to be clear on what really affect the population movement. The missing information is also important to explain whether population movement is an adaptation of different climate components. Are displacements an adaptation process for all climatic drivers? To solve these research gaps, we developed a temporal exponential random graph model (TERGM) in the case of Somali internal displacement. The model was useful to understand how internally displaced person (IDP) flows were formed and dissolved in response to changing climate attributes. In our preliminary result, we found that people were likely to move to destinations with high precipitation when forming new IDP flows. In persistence of IDP flows, they tend to less consider precipitation of their destinations. Therefore, Somali IDPs were adaptive to precipitation in the IDP network formation but not adaptive in the network persistence. This finding improves our understanding of interactions between climate and human displacement.
\end{abstract}

\hypertarget{introduction}{%
\subsection{Introduction}\label{introduction}}

Research questions

\begin{itemize}
\tightlist
\item
  Are population movements climate adaptation processes? For which climatic drivers do we observe adaptation/maladaptation?
\item
  How do attributes at destinations and origins drive displacement? How do they differently work on the formation and persistence of displacement network?
\end{itemize}

\hypertarget{data-and-methods}{%
\subsection{Data and Methods}\label{data-and-methods}}

\hypertarget{results-and-discussions}{%
\subsection{Results and Discussions}\label{results-and-discussions}}

\hypertarget{conclusions}{%
\subsection{Conclusions}\label{conclusions}}

\end{document}
